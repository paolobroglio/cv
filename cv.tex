%%%%%%%%%%%%%%%%%%%%%%%%%%%%%%%%%%%%%%%%%
% "ModernCV" CV and Cover Letter
% LaTeX Template
% Version 1.3 (29/10/16)
%
% This template has been downloaded from:
% http://www.LaTeXTemplates.com
%
% Original author:
% Xavier Danaux (xdanaux@gmail.com) with modifications by:
% Vel (vel@latextemplates.com)
%
% License:
% CC BY-NC-SA 3.0 (http://creativecommons.org/licenses/by-nc-sa/3.0/)
%
% Important note:
% This template requires the moderncv.cls and .sty files to be in the same 
% directory as this .tex file. These files provide the resume style and themes 
% used for structuring the document.
%
%%%%%%%%%%%%%%%%%%%%%%%%%%%%%%%%%%%%%%%%%

%----------------------------------------------------------------------------------------
%	PACKAGES AND OTHER DOCUMENT CONFIGURATIONS
%----------------------------------------------------------------------------------------

\documentclass[11pt,a4paper,sans]{moderncv} % Font sizes: 10, 11, or 12; paper sizes: a4paper, letterpaper, a5paper, legalpaper, executivepaper or landscape; font families: sans or roman

\moderncvstyle{casual} % CV theme - options include: 'casual' (default), 'classic', 'oldstyle' and 'banking'
\moderncvcolor{blue} % CV color - options include: 'blue' (default), 'orange', 'green', 'red', 'purple', 'grey' and 'black'

\usepackage{lipsum} % Used for inserting dummy 'Lorem ipsum' text into the template

\usepackage[scale=0.75]{geometry} % Reduce document margins
%\setlength{\hintscolumnwidth}{3cm} % Uncomment to change the width of the dates column
%\setlength{\makecvtitlenamewidth}{10cm} % For the 'classic' style, uncomment to adjust the width of the space allocated to your name

%----------------------------------------------------------------------------------------
%	NAME AND CONTACT INFORMATION SECTION
%----------------------------------------------------------------------------------------

\firstname{Paolo} % Your first name
\familyname{Broglio} % Your last name

% All information in this block is optional, comment out any lines you don't need
\title{Curriculum Vitae}
\address{Via Giacomo Leopardi, 3 int 7}{Loria, Treviso 31037}
\mobile{(+39) 3400599796}
\email{paolobrogliodev@gmail.com}
%\homepage{staff.org.edu/~jsmith}{staff.org.edu/$\sim$jsmith} % The first argument is the url for the clickable link, the second argument is the url displayed in the template - this allows special characters to be displayed such as the tilde in this example
%\extrainfo{additional information}
\photo[70pt][0.4pt]{pictures/picture} % The first bracket is the picture height, the second is the thickness of the frame around the picture (0pt for no frame)
%\quote{"A witty and playful quotation" - John Smith}

%----------------------------------------------------------------------------------------

\begin{document}

%----------------------------------------------------------------------------------------
%	COVER LETTER
%----------------------------------------------------------------------------------------

% To remove the cover letter, comment out this entire block

%\clearpage

%\recipient{HR Department}{Corporation\\123 Pleasant Lane\\12345 City, State} % Letter recipient
%\date{\today} % Letter date
%\opening{Dear Sir or Madam,} % Opening greeting
%\closing{Sincerely yours,} % Closing phrase
%\enclosure[Attached]{curriculum vit\ae{}} % List of enclosed documents

%\makelettertitle % Print letter title

%\lipsum[1-2] % Dummy text
%\lipsum[4] % Dummy text

%\makeletterclosing % Print letter signature

%\newpage

%----------------------------------------------------------------------------------------
%	CURRICULUM VITAE
%----------------------------------------------------------------------------------------

\makecvtitle % Print the CV title

%----------------------------------------------------------------------------------------
%	EDUCATION SECTION
%----------------------------------------------------------------------------------------

\section{Education}

\cventry{2013}{High School Degree}{Liceo Scientifico Card. C. Baronio}{Vicenza}{\textit{GPA -- 8.0}}{Graduated with 80/100}  % Arguments not required can be left empty
\cventry{2013--2019}{BSc in Computer Science}{University of Padova}{Padova}{\textit{GPA -- 7.5}}{Graduated with 90/100}

\section{Bachelor Thesis}

\cvitem{Title}{\emph{JVM languages alternatives for enterprise development}}
\cvitem{Supervisors}{Professor Mauro Conti}
\cvitem{Description}{
The thesis explored the alternatives to Java in the field of enterprise development. It produced a microservice web application written 
using Kotlin and Spring Boot framework. The service was then deployed to Production environment and used by end users. The thesis was supported 
by Infocert SpA as part of my internship.
}

%----------------------------------------------------------------------------------------
%	WORK EXPERIENCE SECTION
%----------------------------------------------------------------------------------------

\section{Work Experience}

\cventry{2021--Present}{Backend Software Engineer}{\textsc{THRON SpA}}{Piazzola Sul Brenta, Padova, Italy}{}{My job consists of analyzing, designing and developing end users solutions for the Backend side of the company software
\newline{}\newline{}
Detailed projects:
\begin{itemize}
    \item Design and development of the new video conversion system that has been integrated into THRON product. It involved:
        \begin{itemize}
            \item Development of a Scala REST API microservice that enqueues jobs into a queue and persist data to MongoDB
            \item Development of a Scala batch microservice that consumes jobs from a queue
            \item Using AWS Step Functions alongside Python workers on Elastic Container System for file transfer and analysis
            \item Integration with an external provider that performs the actual conversion
            \item Using SQS and SNS for notifications management
        \end{itemize}
    \item Design and development of a workflow service that manages an approval workflow for an end user. It involved the development of:
        \begin{itemize}
            \item A Scala microservice that manages the approval workflow that persist data on MongoDB 
        \end{itemize}
    \item Design and development of the new audio conversion system that has been integrated into THRON product. It involved:
        \begin{itemize}
            \item Development of an AWS Lambda Function that:
                \begin{itemize}
                    \item produced the converted audio
                    \item extracted the audio file cover
                    \item resized the cover according to input resolution
                \end{itemize}
            \item Development of a Scala batch service that started the actual conversion
            \item Using AWS Step Functions for flow and notification management alongside SQS
        \end{itemize}
    \item As part of my role as a Backend Engineer I trained two Junior Backend Engineers into using Scala and being comfortable with company's tools
    \item I produced onboarding documentation for new colleagues
    \item I created pages of documentation in order to help others getting their feet wet with the Backend monolith
    \item I took part in technology stack choices like:
        \begin{itemize}
            \item transitioning to Go and Spring Boot for Java/Scala services
            \item making the company "Cloud-hybrid" by helping evaluating Google Cloud Storage as a replacement for AWS S3
        \end{itemize}
\end{itemize}}

%------------------------------------------------

\cventry{2017--2020}{Backend Software Engineer}{\textsc{Infocert SpA}}{Padova, Italy}{}{My job consists of analyzing, designing and developing end users solutions for the Backend side of the company software
\newline{}\newline{}
Detailed projects:
\begin{itemize}
    \item Development of a gateway integrated with a digital sign server. It involved:
        \begin{itemize}
            \item Development of a Spring Boot HTTP microservice that communicated with a HSM(Hardware Security Module) in order to:
                \begin{itemize}
                    \item generate RSA and ECDSA keypairs
                    \item revoke previously generated private keys
                    \item sign a stream of bytes
                \end{itemize}
            \item Development of a Go GRPC microservice that:
                \begin{itemize}
                    \item implement the PKCS11 protocol that lets the client sign any byte stream and generate keypairs
                \end{itemize}
            \item Integration with the company's Certification Authority
        \end{itemize}
    \item Design and development of a new system for managing electronic invoices. It involved the development of:
        \begin{itemize}
            \item A Quarkus+Java microservice that managed a fair queue of invoices that used Kafka as queue service
            \item The deployment of Kafka and Zookeeper 
        \end{itemize}
    \item Design and development of a new OCSP server. It involved:
        \begin{itemize}
            \item Analysis of OCSP \href{https://datatracker.ietf.org/doc/html/rfc6960}{RFC-6960}:
            \item Development of a Java EE 8 microservice by following RFC specifications
            \item Deployment of the microservice on AWS Elastic Kubernetes Service
            \item Development of a specific feature that involved:
            \begin{itemize}
                \item Data migration from MongoDB to Oracle (the company's main database service)
                \item Usage of RxJava and MongoDB Change Streams in order to react to database changes and perform the data migration in real time
                \item The addition of a REST API that triggers the data migration to avoid data loss
            \end{itemize}
        \end{itemize}
\end{itemize}}

%----------------------------------------------------------------------------------------
%	COMPUTER SKILLS SECTION
%----------------------------------------------------------------------------------------

\section{Computer skills}

\cvitem{Basic}{\textsc{JavaScript}, \textsc{Rust}, \textsc{Elixir}}
\cvitem{Intermediate}{\textsc{Scala}, \textsc{Python}, \textsc{Go} \LaTeX, Linux, Microsoft Windows, MacOS, MongoDB, MySQL, AWS}
\cvitem{Advanced}{\textsc{Java SE 8-11}}

%----------------------------------------------------------------------------------------
%	LANGUAGES SECTION
%----------------------------------------------------------------------------------------

\section{Languages}

\cvitemwithcomment{Italian}{Mothertongue}{}
\cvitemwithcomment{English}{Intermediate}{Talk (B2), Written (B2), Understanding (B2)}

%----------------------------------------------------------------------------------------
%	INTERESTS SECTION
%----------------------------------------------------------------------------------------

\section{Interests}

\renewcommand{\listitemsymbol}{-~} % Changes the symbol used for lists

\cvlistdoubleitem{Drawing}{Hiking}
\cvlistitem{Reading}

%----------------------------------------------------------------------------------------

\end{document}